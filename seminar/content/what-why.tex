\section{Project description}
\begin{frame}[fragile]
  \frametitle{Project description}
  \begin{block}{Motivation}
    \begin{itemize}
      \item Monads are great abstractions for programming with \emph{explicit} effects.\\
            But in general monads do not compose.
      \item Monad Transformers (MT) provide compositionality for monads.\\
            But MTs impose an implicit ordering on effects.
    \end{itemize}
  \end{block}    
  \begin{block}{Problem statement}
    \emph{How may we obtain modular, composable and unordered effects?}
  \end{block}
  \uncover<2->{
    Answer: Handlers for algebraic effects using row polymorphism!}
\uncover<3->{
 \begin{block}{Project aim}
   Provide an implementation of first-class effect handlers in the web-oriented functional programming language Links \cite{Links}.
 \end{block}
}
% \begin{block}{}
%   \begin{columns}
%     \begin{column}{.5\textwidth}
%       Phonebook example using monads
%       \begin{lstlisting}
% askNameAndNumber :: IO (Maybe (String,String))
% askNameAndNumber  = do
%   putStr "Enter your name> "
%   name    <- getLine
%   putStr "Enter your number> "
%   number  <- getLine
%   if isValid number
%     then (*\alert<2->{return \$ Just (name,number)}*)
%     else (*\alert<2->{return \$ Nothing}*)

% isValid :: String -> Bool
% isValid = all isDigit
%       \end{lstlisting}
%     \end{column}
%     \begin{column}{.5\textwidth}
%       Phonebook example using MTs
% \begin{lstlisting}
% askNameAndNumberT :: MaybeT IO (String,String)
% askNameAndNumberT = do
%   (*\alert<3->{lift}*) $ putStr "Enter your name> "
%   name   <- lift getLine
%   (*\alert<3->{lift}*) $ putStr "Enter your number> "
%   number <- getValidNumber
%   return (name,number)
  

% getValidNumber :: MaybeT IO String
% getValidNumber = do num <- lift getLine
%                     guard (isValid num)
%                     return num
% \end{lstlisting}
%     \end{column}
%   \end{columns}
% \end{block}
\end{frame}

% \begin{frame}
%   \frametitle{Project description (II)}
%   \begin{block}{Problem statement}
%     \emph{How may we obtain modular, composable and unordered effects?}
%   \end{block}
%   \uncover<2->{
%     Answer: Handlers for algebraic effects!}
% \uncover<3->{
%  \begin{block}{Project aim}
%    Provide an implementation of first-class effect handlers in the web-oriented functional programming language Links \cite{Links}.
%  \end{block}
% }
% \end{frame}

% \begin{frame}
%  \frametitle{Related work}
%  A handful of implementations already exists:
%  \begin{itemize}
%    \item Eff -- looks and feels like OCaml.
%    \item Haskell libraries, e.g.\small{ \href{http://homepages.inf.ed.ac.uk/slindley/papers/handlers.pdf}{\texttt{effect-handlers}}, \href{http://www.cs.ox.ac.uk/people/nicolas.wu/papers/Scope.pdf}{\texttt{scoped-handlers}}, \href{http://homepages.inf.ed.ac.uk/slindley/papers/frankly-draft-march2014.pdf}{\texttt{Frank}}}.
%    \item Idris library \href{http://eb.host.cs.st-andrews.ac.uk/drafts/effects.pdf}{\small\texttt{Effects}}.
%  \end{itemize}
%  How does our work differ from the aforementioned?
%  \begin{itemize}
%    \item Shortcoming: Limited effect polymorphism.
%    \item We propose, along the lines of Kammar et al. \cite{Kammar2013}, an implementation using row polymorphism.
%  \end{itemize}
% \end{frame}