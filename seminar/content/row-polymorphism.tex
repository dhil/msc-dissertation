\section{Row polymorphism}
% \begin{frame}[fragile]
%   \frametitle{Row polymorphism in two slides}
%   Row polymorphism is a typing discipline for records.
%   \begin{block}{Motivation for row polymorphism}
%     Consider monomorphic records in OCaml:
% \lstset{style=spacey}
% \begin{lstlisting}[basicstyle={\ttfamily\footnotesize}]
% type client = {name: string; id: int}
% type agent  = {name: string; age: int}

% let bob  = {name="Bob"; id=1}
% let alice= {name="Alice"; age=42} 

% let print_name r = print ("Name: " ^ r.name ^ "\n")

% let () = print_name bob; print_name alice
% \end{lstlisting}

% Produces the following error message:
% \begin{verbatim}
% print_name bob;
% ^^^^
% Error: This expression has type client
% but an expression was expected of type agent
% }
% \end{verbatim}
% \end{block}
% \end{frame}

\begin{frame}
  \frametitle{Rho, rho, rho my row}
  \begin{itemize}
    \item Row polymorphism is a typing discipline for records.
  \end{itemize}
\end{frame}
