\section{Related work}\label{sec:relatedwork}
This section discusses and evaluates related work on programming models with handlers and effects.

\subsection{The Eff language}
The \emph{Eff} programming language, by Bauer and Pretnar \cite{Bauer2015}, has a first-class implementation of handlers for algebraic effects. The language has the look and feel of OCaml.
Eff achieves unordered effects through a combination of effect polymorphism and subtyping. 

\subsection{Haskell libraries}
We will discuss two implementations of handlers on top of Haskell by Kammar et. al \cite{Kammar2013} and Wu et. al \cite{Wu2014}.

\subsubsection{Data types á la carte}
Swierstra \cite{Swierstra2008} demonstrates how to compose effectful programs using \emph{free monads} in Haskell.
The free monads form the basis for a framework for encoding handlers and effects which the subsequent libraries use.

\subsubsection{Extensible effects}
A few words on Kiselyov's paper? \cite{Kiselyov2013}.

\subsubsection{Handlers in action}
Kammar et. al considers two different approaches to implement handlers on top of Haskell. One approach is based on free monads \cite{Swierstra2008}, whilst the other is a continuation-based approach \cite{Kammar2013}.

Their handlers are encoded as type classes, thus handlers inherent the limitations of type classes. Particularly, type classes are not first-class in Haskell, so neither are the handlers.
To achieve unordered effects they use type class constraints. Type classes can only be defined in top-level as Haskell do not permit local type-class definition. Consequently, every effect handler must be defined in the top-level too.

Furthermore, the order in which handlers are composed leak into the type signature, because their (open) handlers explicitly mention a parent handler \cite{Kammar2013}. Albeit, it does not cause an issue like Monad Transformer ordering issue, but it is still undesirable.

Kammar et. al hypothesises that an implementation based on row polymorphism may remedy the limitations and yield a cleaner design \cite{Kammar2013}.

\subsubsection{Handlers in scope}
Wu et. al investigate how to use handlers to delimit the scope of effects \cite{Wu2014} as using handlers for scoping has limitations. In other words, the ordering of handlers may affect the semantics. 

They present two solutions embedded in Haskell \cite{Wu2014}. The first solution extends the existing effect handler framework based on free monads with so-called \emph{scope markers} which fits nicely into the framework. However they demonstrate that handlers along with scope markers are insufficient to capture higher-order scoped constructs properly.

Their second approach is continuation-based and provides a \emph{higher-order syntax} that allows to embed programs with scoping constructs \cite{Wu2014}. However it remains an open question whether their implementation is viable in other languages than Haskell.

\subsection{Frank}
The Frank programming language by McBride \cite{McBride2014} takes the notion of effect handlers to the extreme. In Frank there are no functions, there are only handlers. Moreover, it employs an interesting evaluation order known as ``call-by-push-value'' (CBPV). Intuitively, CBPV is the unification of the strict call-by-value and non-strict call-by-name semantics. The choice whether to employ the strict or non-strict semantics has been made explicit to the programmer.

Frank distinguishes between computations and values as a consequence side-effects can only occur in computations. Hence there is a clear separation between segments of code where effects might occur and where effects are guaranteed never to occur.

%In contrast to Eff, and the Haskell libraries, it provides its own syntax, and thus does not try to adapt to an existing language.

\subsection{Idris' Effects}
Brady \cite{Brady2013} presents the library {\scshape{Effects}} for the dependently-typed, functional programming language {\scshape{Idris}}.

\subsection{Koka with row polymorphic effects}
Leijen's programming language Koka is an effect-based web-oriented language \cite{Leijen2014}. 
It supports arbitrary user-defined effects \cite{Vazou2015}.
Notably, Koka uses row polymorphism to capture unordered effects however Koka's row polymorphism allow duplicate effect occurrences which stands in contrast to the approach we propose.
In particular, Koka has no notion of effect handler except for exception handlers which are to some extent reminiscent of those in Java, C\#, etc.