\chapter{Introduction}
Programs are effectful, i.e. they may operate on some state, divide by zero, halt prematurely.
Imperative programs are inherently effectful as computations perform a series of implicit effects on a shared global state.
Most imperative programming languages are not \emph{explicit} effects, even though, effects underpin their foundational assumption \cite{Meijer2014}.
In contrast pure functional programming languages often use monads as a basis for effectful programming. Unfortunately, monads impose an ordering on effects which impairs compositionality and modularity \cite{Kammar2013}.

Plotkin and Pretnar's \emph{handlers for algebraic effects} \cite{Plotkin2013} provide an alternative to monads, where effectful computations are composed from abstract operations which are interpreted by handlers. In this thesis, we join handlers and algebraic effects with \emph{row polymorphism}. The result is a compelling programming model which unifies compositionality, modularity and explicit effectful programming. We implement and examine our programming model in the functional programming language Links.

The remainder of the chapter analyses the problem with monads as a basis for effectful programming.
%A recipe for the ideal programming model would include: Compositionality, modularity and explicit effects.
%Compositionality lets us break a complex problem into smaller constituent problems. The complexity of a greater problem can be harnessed by composing solutions to smaller, likely easier, constituent problems. Moreover, compositionality encourage reuse of specialised components to solve future problems.
%Modularity refers to the degree of coupling between components. A high degree of modularity implies low coupling between components. Low coupling can be achieved by keeping interfaces between connected components abstract. Abstract interfaces lets us exchange one concrete implementation for another implementation effortlessly.
%Together modularity and compositionality form the basis for a powerful programming model. However, being explicit about effects is often neglected \cite{Meijer2014}. An effect give a static description of the possible state-changing actions that may occur during evaluation of a particular piece of code. Moreover, effects can be informative for the compiler as well as the programmer \cite{Kammar2012,Brady2013,Meijer2014}.

