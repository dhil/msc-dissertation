\section{Relative performance}\label{sec:eval-performance}
Since Links is an interpreted language it does not make sense to measure the raw execution speed of handled computations as the overhead incurred by the interpreter is likely to be dominant. Instead, we will measure the relative cost incurred by using handlers.

\subsection{Benchmarks setup}
The experiments were conducted on a standard Informatics DICE Machine\footnote{Machine name: Enna. Specifications: Intel Core i5-4570 3.20 Ghz, 8 GB Ram, Scientific Linux 6.6 (Carbon) running Linux kernel 2.6.32-504.16.2.el6.x86\_64}. The following three different micro benchmarks were used:
\begin{itemize}
  \item Stateful counting: Counting down from $10^7$ to $0$ using a closed state handler.
  \item Stateful counting with logging: Counting down from $10^7$ to $0$ using the state logging handler from Example \ref{ex:logging-state}.
  \item Nim game tree generation: Generation of game tree with starting configuration $n = 20$ using the handler from Example \ref{ex:nim-game-tree}.
\end{itemize}
Each handler program has two pure counterparts. For the stateful counting benchmarks the first pure version is a direct, tail-recursive implementation which passes the current state as an explicit parameter between invocation. The second pure implementation encapsulates state inside a function in similar fashion to the state handler. 

The first pure game tree generator program is a hard-coded, direct implementation for the specific restricted version of Nim we used in Section \ref{sec:interpreting-nim}. The second pure version is more general, and will generate the game tree under any rules. Thus it resembles the handler version. The generality is achieved through use of higher-order functions such as \code{map}, \code{zip}, etc.
%Each benchmark program has a pure counterpart. For instance pure stateful counting passes the state as a parameter to the counting function. The pure Nim game tree generator is hard-coded to produce game trees under the restrictions explained in Section \ref{sec:interpreting-nim}. The source code for each pure program is listed in Appendix \ref{app:bench-pure}.

%For each benchmark we take 10 samples. To eliminate noise caused by programs benefiting from cache locality the sampling has been interleaved. That is, first we run a benchmark once to produce one sample, then we run another benchmark to produce one sample, and so forth. This process has been repeated 10 times to produce 10 samples for each benchmark.

The built-in performance-measuring mechanism in the Links interpreter has been used to measure the execution time. The execution time only includes the run time of the program, that is it does \emph{not} include loading up the Links interpreter or program compilation.
\subsection{Results}
Table \ref{tbl:stateful-counting} displays the results obtained from the first stateful counting benchmark. The direct, tail-recursive implementation is twice the alternative implementations. A state change does not incur an extra cost, because it is passed as an explicit parameter between invocations. However, the handler implementation and the second pure implementation both use functions to encapsulate state, therefore each state change causes a new function allocation. The results would suggest that the penalty is high for repetitive allocation of functions.
\begin{table}[H]
  \centering
  \begin{tabular}{| l | r | r |}
    \cline{2-3}
    \multicolumn{1}{c |}{} & \multicolumn{1}{c |}{Time (ms)} & {Relative speed} \\
    \hline
    Pure I, tail-recursive &  9629.14 & 1.0 \\
    \hline
    Pure II, function state &  20364.6 & 0.47 \\
    \hline
    Closed handler       &  14406.11 & 0.5 \\
    \hline
  \end{tabular}\caption{Results obtained from the stateful counting benchmark.}\label{tbl:stateful-counting}
\end{table}
The same seems to be evident for the second stateful counting benchmark. The results are shown in Table \ref{tbl:stateful-counting-logging}. The tail-recursive pure implementation is significantly faster than the two alternative implementations. It is roughly $3.5$ times faster than the pure implementation that encapsulates state inside functions, and further it is about $8$ times faster than the handler version. However, this time there is a big difference between the the second pure implementation and the handler implementation. The pure version is little less than twice as fast as the handler version. Possibly, the additional cost is incurred by the handler being open. Each change in state causes two additional operations to be discharged. In addition the handler stack is unwound three times. In particular, three different delimited continuations are invoked during one state change. When a continuation returns control to a handler, it implicitly passes through the return-cases of that particular handler's predecessors (in order to lift the result of the computation). So, the open handler stack incurs a large extra cost.
\begin{table}[H]
  \centering
  \begin{tabular}{| l | r | r |}
    \cline{2-3}
    \multicolumn{1}{c |}{} & \multicolumn{1}{c |}{Time (ms)} & {Relative speed} \\
    \hline
    Pure I, tail-recursive &  19097.1 & 1.0 \\
    \hline
    Pure II, function state &  68615.1 & 0.28 \\
    \hline
    Open handler (stack size: 4) &  161458.15 & 0.12 \\
    \hline
  \end{tabular}\caption{Results obtained from the stateful counting with logging benchmark.}\label{tbl:stateful-counting-logging}
\end{table}
Finally, in the Nim game tree generator benchmark the hard-coded pure version is superior in terms of execution speed. The handler implementation and generic pure implementation perform, respectively, at 6\% and 7\% of the hard-coded version. The results are shown in table \ref{tbl:nim-bench}.
The generic pure implementation is about 14\% faster than the handler implementation.
\begin{table}[H]
  \centering
  \begin{tabular}{| l | r | r |}
    \cline{2-3}
    \multicolumn{1}{c |}{} & \multicolumn{1}{c |}{Time (ms)} & {Relative speed} \\
    \hline
    Pure I, hard-coded &  814.98 & 1.0 \\
    \hline
    Pure II, generic   &  12441.01 & 0.07 \\
    \hline
    Closed handler     &  14406.11 & 0.06 \\
    \hline
  \end{tabular}\caption{Results obtained from the Nim game tree generation benchmark.}\label{tbl:nim-bench}
\end{table}
These results suggest that there is plenty of room for optimisations. Kammar et al. achieve better relative performance figures than us \cite{Kammar2013} with their embedding of handlers in Haskell. However their handlers desugar into monads which the Haskell compiler extensively optimises. Thus, it is likely that with optimisations with could achieve far better performance figures.