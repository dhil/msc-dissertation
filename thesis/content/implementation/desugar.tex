
\section{Early desugaring of handlers}
The \code{handler} and \code{open handler} constructs are syntactic sugar. They get desugared into a legacy construct from an early implementation. The initial implementation used a \code{handle}-construct for handlers. Figure \ref{fig:closedhandler-desugar} displays the conceptual transformation of \code{handler} to \code{handle}. 
This desugaring takes place right after the parsing phase. The early desugaring is beneficial because it allows us to take full advantage of the earlier implementation, whilst providing a more convenient syntax for handlers.
\begin{figure}[h]
    \centering
    \begin{subfigure}[c]{0.45\textwidth}
        \centering
\begin{lstlisting}[style=links]
handler(m) {
  case Op§*$_i$*§(p§*$_i$*§,k§*$_i$*§) -> b§*$_i$*§
  case Return(x) -> b
}
\end{lstlisting}
    \end{subfigure}%
    ~ 
    \begin{subfigure}[c]{0.1\textwidth}
      $\Rightarrow$
    \end{subfigure}%
    ~
    \begin{subfigure}[c]{0.45\textwidth}
        \centering
\begin{lstlisting}[style=links]
fun(m) {
  handle(m) {
    case Op§*$_i$*§(p§*$_i$*§,k§*$_i$*§) -> b§*$_i$*§
    case Return(x) -> b
  }
}
\end{lstlisting}
    \end{subfigure}
\caption{The \code{handler}-construct gets desugared into a \code{handle}-construct where the computation $m$ is abstracted over using a function.}\label{fig:closedhandler-desugar}
\end{figure}

The \code{open handler}-constructs get desugared in a similar fashion, but, with a small twist: The \code{handle}-construct gets wrapped inside a thunk. The extra layer of indirection entailed by this transformation is \emph{the key} to make handlers composable. The crucial insight is that by transforming every open handler into a thunk compositionality follows for free because computations are modelled as thunks.
Figure \ref{fig:openhandler-desugar} shows the conceptual transformation for \code{open handler}-constructs.

\begin{figure}[h]
    \centering
    \begin{subfigure}[c]{0.45\textwidth}
        \centering
\begin{lstlisting}[style=links]
open handler(m) {
  case Op§*$_i$*§(p§*$_i$*§,k§*$_i$*§) -> b§*$_i$*§
  case Return(x) -> b
}
\end{lstlisting}        
    \end{subfigure}%
    ~ 
    \begin{subfigure}[c]{0.1\textwidth}
      $\Rightarrow$
    \end{subfigure}%
    ~
    \begin{subfigure}[c]{0.45\textwidth}
        \centering
\begin{lstlisting}[style=links]
fun(m) {
  fun() {
    handle(m) {
      case Op§*$_i$*§(p§*$_i$*§,k§*$_i$*§) -> b§*$_i$*§
      case Return(x) -> b
    }
  }
}
\end{lstlisting}       
    \end{subfigure}
\caption{The \code{open handler}-construct gets desugared into a thunked \code{handle}-construct.}\label{fig:openhandler-desugar}
\end{figure}