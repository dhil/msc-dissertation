\section{Open handlers}\label{sec:openhandlers}
Open handlers are the dual to closed handlers when we think in terms of bounds on effects. An open handler give a lower bound on the kind of effects it will handle. Through composition of open handlers we can achieve a tighter bound on the handled effects. Consequently, one can delegate responsibility to \emph{specialised} handlers that handle a particular subset of the effects. Unhandled operations are forwarded to subsequent handlers. In other words, an open handler partially interprets an abstract computation and leaves the remainder for other handlers.

In Links the concrete syntax for open handlers is similar to that for closed handlers. To declare an open handler one simply uses the keyword \code{open} in the declaration, e.g.
\begin{lstlisting}[style=links]
open handler h(m) {
  case Op§*$_i$*§(p§*$_i$*§,k§*$_i$*§)  -> body§*$_i$*§
  case Return(x)  -> body
}
\end{lstlisting}
The inferred type for the open handler \code{h} is more complex than its closed counterpart:
\[ \code{h} : (() \xrightarrow{ \{ \type{Op}_i: a_i \to b_i \; | \; \rho\}\; } c) \to () \xrightarrow{ \{ \type{Op}_i: \alpha_i \; | \; \rho\}\;} d\]
Notice that the effect row of the input computation is \emph{polymorphic} as signified by the presence of the row variable $\rho$. Accordingly, the input computation may perform more operations than the handler handles. The output type of an open handler looks very similar to its input type. The input as well as the output is a thunk. Moreover, their effect rows share the same polymorphic row variable $\rho$. But their operation signatures differ. The polymorphic variable $\alpha_i$ denotes that the $i$'th operation may be present or absent from the effect row.

Since the input type and output type of open handlers match we can compose open handlers seamlessly. The order of composition implicitly defines a stack of handlers. For example the composition of three handlers $(h_1 \circ h_2 \circ h_3)(m)$ applied to some computation $m$ defines a stack where $h_3$ is the top-most element. Thus the handler stack is built outside in. The ordering inside the stack determines which handler is invoked when $m$ discharges an operation. First the top-most handler is invoked, and if it cannot handle the discharged operation then the operation is forwarded to the second top-most handler and so forth.

Consequently, the order of composition may affect the semantics, say, $h_1$ and $h_2$ interpret the same operation differently, then, $h_1 \circ h_2$ and $h_2 \circ h_1$ potentially yield different results.

The composition of open handlers is itself an open handler, thus it will return a thunk itself. For example $(h_1 \circ h_2 \circ h_3)(m)$ yield some nullary function $() \to a$ which we must explicitly invoke to obtain the result of the computation $m$. To avoid this extra invocation recall the \code{force} handler from Section \ref{sec:transform}. We can apply the closed handler \code{force} to obtain the result of $m$ directly, e.g. $(\code{force} \circ h_1 \circ h_2 \circ h_3)(m)$ yields a result of type $a$ immediately.

\subsection{An effectful coffee dispenser in Links}
In Section \ref{sec:problem-with-monads} and \ref{sec:mt} we implemented a model of a coffee dispenser in Haskell using monads (Examples \ref{ex:coffee1} and \ref{ex:coffee2}). However, it was difficult to extend the model to include more properties like writing to a display and system failures without resorting to Monad Transformers due to regular monads' lack of compositionality.

In contrast, the modularity and compositionality afforded by (open) handlers enable us to easily implement a the a highly modular coffee dispenser model in Links. Example \ref{ex:coffee-links} implements the model. % and Example \ref{ex:coffee-links2} demonstrates how the modularity and compositionality of handlers extend the model without breaking a sweat.
\begin{example}[Coffee dispenser]\label{ex:coffee-links}
The coffee dispenser performs two operations directly
\begin{enumerate}
  \item \type{Ask}: Retrieves the inventory.
  \item \type{Tell}: Writes a description of an item to some medium.
\end{enumerate}
Indirectly, the coffee dispenser may perform the \type{Fail} operation when it looks up an item. Thus the type of the dispenser is 
\[ \code{dispenser} : a \xrightarrow{\{\type{Ask}:() \to [(a, b)], \type{Fail}:() \to b,\type{Tell}:b \to c| \rho\}\;} c \]
We compose the coffee dispenser from the aforementioned operations and the look-up function, e.g.
\begin{lstlisting}[style=links]
fun dispenser(n) {
  var inv = do Ask();
  var item = lookup(n,inv);
  do Tell(item)
}
\end{lstlisting}
The monadic coffee dispenser model used three monads: \type{Reader}, \type{Writer} and \type{Maybe} to model the desired behaviour. We will implement three handlers which resemble the monads. First, let us implement \type{Reader}-monad as the handler \code{reader} whose type is 
\[  (() \xrightarrow{\{\type{Ask}:(a) \to [(\type{Int}, [|\type{Coffee}|\type{Tea}|\rho_1|])] \, | \,\rho_2\}\;} b) \to () \xrightarrow{\{\type{Ask}:\alpha|\rho_2\}\;} b \]
For simplicity we hard-code the inventory into the handler
\begin{lstlisting}[style=links]
open handler reader(m) {
  case Ask(_,k)  -> k([(1,Coffee),(2,Tea)])
  case Return(x) -> x
}
\end{lstlisting}
When handling the operation \type{Ask} the handler simply invokes the continuation \code{k} with the inventory as parameter. Like in Example \ref{ex:coffee1} we model the inventory as an association list.

Second, we implement the handler \code{writer} which provide capabilities to write to a medium. We let the medium be a regular string. The handler's type is
\[  (() \xrightarrow{\{\type{Tell}: [|\type{Coffee}|\type{Tea}|] \to \type{String} \, | \,\rho\}\;} a) \to () \xrightarrow{\{\type{Tell}:\alpha|\rho_2\}\;} a \]
and its definition is
\begin{lstlisting}[style=links]
open handler writer(m) {
  case Tell(Coffee,k) -> k("Coffee")
  case Tell(Tea,k)    -> k("Tea")
  case Return(x)      -> x
}
\end{lstlisting}
Here we use pattern-matching to convert \type{Coffee} and \type{Tea} into their respective string representations.

Finally, we implement the \code{lookup} function which given a key and an association list returns the element associated with the key if the key exists in the list, otherwise it discharges the \type{Fail}-operation to signal failure
\begin{lstlisting}[style=links]
fun lookup(n, xs) {
  switch (xs) {
    case [] -> do Fail()
    case ((k, e) :: xs) -> if (n == k) { e }
                          else { lookup(n, xs) }
  }
}
\end{lstlisting}
To handle failure we reuse the \type{maybe}-handler from Section \ref{sec:maybehandler} with the slight change that we make it an open handler. Now, we just have to glue all the components together
\begin{lstlisting}[style=links]
fun runDispenser(n) {
  force(maybe(writer(reader(fun() { dispenser(n) }))))
}
\end{lstlisting}
Note, that in this example the order in which we compose handlers is irrelevant.
Running a few examples we see that it behaves similarly to the monadic version we implemented in Section \ref{sec:mt}
\begin{lstlisting}[style=links]
links> runDispenser(1)
Just("Coffee") : [|Just:String|Nothing|§*$\rho$*§|]

links> runDispenser(2)
Just("Tea") : [|Just:String|Nothing|§*$\rho$*§|]

links> runDispenser(3)
Nothing() : [|Just:String|Nothing|§*$\rho$*§|]
\end{lstlisting}
\end{example}
Observe that when we implemented the monadic version of the \code{dispenser} using Monad Transformers we had to pay careful attention to the ordering of effects up front because we had to lift certain operations. This issue is no longer present with handlers. In fact, we first defined \code{dispenser} without considering the concrete the interpretation of the operations \type{Ask} and \type{Tell} (and \type{Fail}). Furthermore, the effect ordering does not leak into the inferred effect row as opposed to Monad Transformers. The effect row typing is pivotal to the modular design afforded by handlers. Programmers can truly implement composable components independently as they do not have to worry about issues such as the shadow issue which is caused by having an ordering on effects.

\subsection{Reinterpreting Nim}\label{sec:reinterpreting-nim}
In Section \ref{sec:interpreting-nim} we gave various interpretations of the game Nim using closed handlers. Example \ref{ex:nim-cheat-detection} demonstrates how we can use the compositionality of open handlers to extend the game with an additional cheat detection mechanism without breaking a sweat.

We reuse the game model and auxiliary functions from Section \ref{sec:reinterpreting-nim}.
\begin{example}[Cheat detection in Nim]\label{ex:nim-cheat-detection}
First, we implement a function that given a player, the number of remaining sticks $n$ and the number, and a continuation $k$ determines whether the player cheats. We call this function \code{checkChoice}, it will perform two operations: \type{Move} and \type{Cheat}, the former simulates a particular move whilst the latter operation is used to signal that cheating has occurred. The type of the function is
\[
\code{checkChoice} : (a, b, \type{Int} \xrightarrow{E} c) \xrightarrow{E} c
\] 
where $E \defas \{\type{Cheat}:(a, \type{Int}) \to c,\type{Move}:(a, b) \to \type{Int}|\rho\}$. The following is its implementation:
\begin{lstlisting}[style=links]
fun checkChoice(player,n,k) {
  var take = do Move(player,n);
  if (take < 1 || 3 < take) { # Cheater detected!
    do Cheat(player,take)
  } else {                    # Otherwise OK
    k(take)
  }
}
\end{lstlisting}
First, we simulate the player's move. If the player's choice is not in the set of valid moves $\{1,2,3\}$ then the function signals that cheating has occurred, otherwise the continuation \code{k} is invoked to actually perform the move. Now, it is straightforward to implement a handler which uses \code{checkChoice} to detect cheating, e.g.
\begin{lstlisting}[style=links]
open handler checkgame(m) {
  case Move((player,n),k) -> checkChoice(player,n,k)
  case Return(x)          -> x
}
\end{lstlisting}
Note that the type of \code{checkgame} is $(() \xrightarrow{E} c) \to () \xrightarrow{E} c$ where the effect row $E$ is the same as above. Hence \code{checkgame} is itself an abstract computation.
Therefore we will need two more handlers which interpret the \type{Cheat} and \type{Move} operations. We encode the cheater's strategy into the handler which handles the additional \type{Move} operation discharged by \code{checkgame}, e.g.
\begin{lstlisting}[style=links]
fun cheater(n,k) {
    k(n)
}

open handler aliceCheats(m) {
  case Move((Alice,n),k) -> cheater(n,k)
  case Move((Bob,n),k)   -> perfect(n,k)
  case Return(x)         -> x
}
\end{lstlisting}
Here a cheater's strategy is simply to take all sticks in the heap and thereby win the game in one single move. In the handler \code{aliceCheats} we assign the cheater's strategy to Alice whilst Bob plays the perfect strategy. Thus if we play without cheat detection then Alice will always win in a single move because she always starts.

Finally, we interpret the \type{Cheat} operation by halting the game and reporting the cheater, e.g.
\begin{lstlisting}[style=links]
open handler cheatReport(m) {
  case Cheat((Alice,n),k) -> error("Cheater Alice took " ^^ intToString(n) ^^ " sticks")
  case Cheat((Bob,n),k)   -> error("Cheater Bob took " ^^ intToString(n) ^^ " sticks")
  case Return(x)          -> x
}
\end{lstlisting}
Here, we pattern match on the player to determine who cheated. The \code{error} function halts the game and reports to standard out who cheated along with how many sticks the player took. Now, we can put everything together and try a few examples:
\begin{lstlisting}[style=links]
fun checkedGame(m) {
  force(aliceCheats(cheatReport(checkgame(m))))
}

links> play(checkedGame, 36);
*** Fatal error : Cheater Alice took 36 sticks

links> play(checkedGame, 3);
Alice() : [|Alice|Bob|§*$\rho$*§|]
\end{lstlisting}
Alice still wins when $0 < n \leq 3$ because in this particular game configuration it is a legal move to take all sticks.
Moreover, observe that the order in which we compose handlers is important in this example because \code{checkgame} is itself an abstract computation, therefore if we swap \code{aliceCheats} and \code{checkgame} the cheat detection mechanism never gets invoked. Accordingly, Alice would always win because she cheats.
\end{example}
Like in the previous Nim game examples we changed the strategic behaviour of the players without changing the game model (\code{aliceTurn} and \code{bobTurn}), however, in addition in Example \ref{ex:nim-cheat-detection} we also extended the game mechanics without changing the game model.

\subsection{Simulating state}\label{sec:state}
We can use handlers to simulate stateful computations, and thereby enabling us to abstract over how state is interpreted. To handle state we need two operations:
\begin{itemize}
  \item $\type{Get}:() \to s$ that reads the current state.
  \item $\type{Put}:s \to ()$ that updates the state.
\end{itemize}
We will use the following simple stateful computation to illustrate stateful interpretations:
\begin{lstlisting}[style=links]
fun scomp() {
  var s = do Get(); do Put(s + 2);
  var s = do Get(); do Put(s + s);
  do Get()
}
\end{lstlisting}
First, the computation reads the current integer state, then the state is incremented by 2. The new state is then read and doubled. The computation returns the final state. Example \ref{ex:state} gives a direct interpretation of state.
\begin{example}[State handler]\label{ex:state}
Since Links does not have mutable variables we have to find another way to implement state. A pure functional approach is to pass the state around as an explicit parameter.
Basically, we adopt this approach to implement state, however, we will introduce an extra layer of indirection
to pass the state around. We will abstract over state by encapsulating it inside a function. The function will take a concrete state $s$ as input parameter. For the \type{Get}-, \type{Put}- and \type{Return}-cases the handler returns a new state-encapsulating function. The \code{state} handler is defined as follows
\begin{lstlisting}[style=links]
open handler state(m) {
  case Get(_,k)  -> fun(s) { k(s)(s)  }
  case Put(p,k)  -> fun(s) { k(())(p) }
  case Return(x) -> fun(s) { x }
}
\end{lstlisting}
The \code{state} handler is partially lazy as when either \type{Get} and \type{Put} are discharged the handler returns a single parameter function. Therefore the handler basically suspends the handled computation first time an operation is discharged.

At first glance the \code{state} handler may seem dubious. Essentially, the handler builds a chain of functions which passes the state around. Let us break it down: The state function returned by the \type{Get}-case invokes the continuation with the current state which is substituted at the invocation site of \type{Get}. The continuation returns the next state function to which the same state is passed.
The \type{Put}-case invokes the continuation with unit and passes the modified state \code{p} as input to the next state function.
When a stateful computation finishes the \type{Return}-case lifts the result into a function, that ignores its argument. The type of the \code{state} handler is 
\[ 
(() \xrightarrow{ \{ \type{Get}: () \to s, \type{Put}: s \to () \; | \; \rho\}\;} a) \to () \xrightarrow{ \{\type{Get}:\theta_1, \type{Put}:\theta_2 \; | \; \rho\}\;} s \to a 
\]
where the type $s \to a$ is the type of a state function. The type variable $s$ is the type of the initial state. In order to execute a stateful computation it is convenient to have a driver function \code{runState} which abstracts away these details, e.g.
\begin{lstlisting}[style=links]
fun runState(s0, m) {
  force(state(m))(s0)
}
\end{lstlisting}
Applying \code{runState} to some initial state $s_0$ and \code{scomp} we see that it behaves as desired:
\begin{lstlisting}[style=links]
links> runState(0, scomp);
2 : Int
links> runState(-2, scomp);
-2 : Int
links> runState(3, scomp);
8 : Int
\end{lstlisting}
\end{example}
The \code{state} handler in Example \ref{ex:state} returns the most recent state. Incidentally, taking advantage of the composition, we can give a different interpretation which track state changes as Example \ref{ex:state-tracking} demonstrates. 
\begin{example}[Stateful logging]\label{ex:state-tracking}
We extend the \code{state} handler with a logging capability, however, we will not add this capability directly to the handler. Instead we are going to use composition to construct a stateful handler which keeps track of state changes. The idea is to introduce a new operation $\type{LogPut} : s \to ()$ which logs some state of type $s$. Further, we introduce two new handlers:
\begin{itemize}
  \item $\code{putLogger} : (() \xrightarrow{ \{\type{LogPut}: s \to (), \type{Put}:s \to () \; | \; \rho_1\}\;} s) \to () \xrightarrow{ \{ \type{LogPut}: s \to (),\type{Put}:s \to () \; | \; \rho_1\}\;} s$
  \item $\code{logState} : (() \xrightarrow{\{ \type{LogPut}: s \to () \; | \; \rho_2\}\;} s) \to () \xrightarrow{\{\type{LogPut}:\theta \; | \; \rho_2 \}\;} (s, [s])$
\end{itemize}
The \code{putLogger} handler handles a \type{Put} operation, however, it discharge a \type{Put}-operation itself along with a \type{LogPut}-operation as signified by their presence in the output effect signature.
We implement \code{putLogger} as follows
\begin{lstlisting}[style=links]
open handler putLogger(m) {
  case Put(p,k)  -> { do LogPut(p); do Put(p); k(()) }
  case Return(x) -> x
}
\end{lstlisting}
In the \type{Put}-case the handler first discharges \type{LogPut} to log the state change, and then it performs the actual state change by discharging \type{Put}. In some sense \type{putLogger} acts as a ``middleman'' because it relies wholly on other handlers to interpret \type{LogPut} and \type{Put}.

The \code{logState} handler builds the log, e.g.
\begin{lstlisting}[style=links]
open handler logState(m) {
  case LogPut(x,k) -> { 
        var s = k(());
        var xs = second(s);
        (first(p), (x :: xs))
      }
  case Return(x) -> (x, [])			 
}
\end{lstlisting}
The handler returns the final state along with a list of previous state changes. In the \code{LogPut}-case the handler first invokes the continuation in order to advance the stateful computation. The continuation returns a pair which contains the final state along with a list of changes. The handler preserves the first component, but it extends the second component with the previous state \code{x}.
In order to handle \type{Get} and \type{Put} we compose the above handlers with the \code{state} handler. Finally, we can reinterpret the computation \code{scomp}:
\begin{lstlisting}[style=links]
links> runState(0, scomp);
(2, [1, 2]) : (Int, [Int])
links> runState(-2, scomp);
(-2, [-1, -2]) : (Int, [Int])
links> runState(3, scomp);
(8, [4, 8]) : (Int, [Int])
\end{lstlisting}
\end{example}
\subsection{A handler based parsing framework}\label{sec:parserlibrary}
The state handler enable us to implement abstract, stateful handlers that employ the state operations. In this section we will demonstrate how to implement a simple, but highly modular, backtracking parser as a handler that interprets parser combinators. The result is a small parser framework in Links.

To demonstrate the framework we implement a parser for the small language $\text{\scshape{Palindromes}} \defas \{ w \in \{a,b\}^* \; | \; w\text{ is a palindrome}\}$. The language {\scshape{Palindromes}} is generated by the following grammar
\begin{equation}\label{eq:grammar}
  P ::= a \; | \; b \; | \; aPa \; | \; bPb \; | \; \varepsilon
\end{equation}
The grammar is simple, yet it contains choice, concatenation, recursion and the empty string ($\varepsilon$) which are sufficient to make an interesting example.

\subsubsection{Parser combinators}
The principal idea behind parser combinators is to compose parsers from smaller, simpler parsers. The parsers are abstract computations composed using the following three operations:
\begin{itemize}
  \item $\type{Choose} : () \to \type{Bool}$ that makes a nondeterministic choice.
  \item $\type{Token}  : () \to \type{Char}$ that consumes a character from the input stream.
  \item $\type{Fail}   : () \to ()$ that signals failure.
\end{itemize}
The parsers are implemented as functions whose types are $() \xrightarrow{E} ()$, where $E$ is an open effect row that contains either all, some or none of the above operations. In other words, a parser is a nullary function which may cause several effects, and returns unit. From its signature it is not clear, that a parser does anything sensible. In fact, the purpose of parsers is to capture the structure of a grammar, and we leave the rest of the work to a handler. 

The two simplest parsers are \code{empty} and \code{char} which accepts the empty string and one particular character, respectively. Their definitions are given below:
\begin{figure*}[h]
    \centering
\begin{tabular}{c|c}
Empty string parser & Single character parser \\
\hline
    \begin{subfigure}[c]{0.35\textwidth}
        \centering
\begin{lstlisting}[style=links]
      fun empty() {
        ()
      }
\end{lstlisting}
    \end{subfigure}%
    &
    \begin{subfigure}[c]{0.36\textwidth}
        \centering
\begin{lstlisting}[style=links]
  fun char(c) {
   fun() {
    var t = do Token();
    if (t == c) { () }
    else { do Fail() }
  }}
\end{lstlisting}
    \end{subfigure}
\end{tabular}
\end{figure*}
The parser \code{empty} does nothing, it simply returns unit. The function \code{char} is not really a parser, but rather a \emph{parser generator}. It takes a character \code{c} as input, and generates a parser that checks whether \code{c} is the next character in the input stream. If \code{c} is the next character, then the parser returns unit otherwise it signals failure.

We require two more basic parser generators: Choice and sequence. The choice generator takes two parsers as input, and makes a nondeterministic choice between the two. The sequence generator takes two parsers, and applies them in sequence. For syntactic conveniency, we define them as binary operators in Links:
\begin{figure*}[h]
    \centering
\begin{tabular}{c | c}
Choice parser & Sequence parser \\
\hline
    \begin{subfigure}[c]{0.35\textwidth}
        \centering
\begin{lstlisting}[style=links]
  op p <|> q {
    fun() {
      if (do Choose()) { p() }
      else { q() }
    }
  }
\end{lstlisting}
    \end{subfigure}%
    &
    \begin{subfigure}[c]{0.36\textwidth}%
        \centering
\begin{lstlisting}[style=links]
  op p <*> q {
    fun() {
      p(); q()
    }
  }
\end{lstlisting}
    \end{subfigure}
\end{tabular}
\end{figure*}

Choice (\code{<|>}) generates a parser which discharges \type{Choose} to decide whether to apply parser \code{p} or \code{q}. Sequence (\code{<*>}) generates a parser which applies parsers \code{p} and \code{q} in sequence.

Later, Example \ref{ex:palindromes} demonstrates how to compose these four parsers to construct a parser for {\scshape{Palindromes}}.

\subsubsection{Interpreting parsers}
The previous section gave the building blocks for constructing parsers. In this section we will implement an abstract, stateful handler which interprets parsers. The handler has to handle the three operations \code{Choose}, \code{Token} and \code{Fail}. Furthermore, it will use the state operations \code{Get} and \code{Put} to manipulate the parsing state. The parsing state is a pair $\type{PState} \defas ([\type{Char}],[\type{Char}])$ where the first component contains parsed symbols, and the second component contains the remainder of the input stream. Furthermore, if the input string is not in the language then the \code{parser}-handler should produce an error, otherwise it should return the parsed string. Thus, the \code{parser}-handler has the rather involved type
\begin{align*}
&(() \xrightarrow{\{\type{Choose}: () \to \type{Bool},\type{Fail}: a_1 \to a_2,\type{Get}: () \to \type{PState},\type{Put}: \type{PState} \to (), \type{Token}: () \to \type{Char} \; | \; \rho\}\;} b)\\
&\to () \xrightarrow{\{\type{Choose}: \theta_1,\type{Fail}: \theta_2,\type{Get}: () \to \type{PState},\type{Put}: \type{PState} \to (), \type{Token}: \theta_3 \; | \; \rho\}\;} \type{Maybe}(\type{[Char]})
\end{align*}
This type witnesses a cosmetic issue with the implementation, because effects are not explicitly named the effect signature easily blows up and therefore becomes difficult to read. The parser-handler is more involved than the previous handlers we implemented. Therefore, we name the handler \code{parser} and implement it one case at a time. The easiest case \type{Fail} which, unconditionally, returns \code{Nothing}. We will not show it here, instead we begin with the case for \type{Token}:
\begin{lstlisting}[style=links]
case Token(_,k) -> {
  var s = do Get();
  var stream = second(s);
  switch (stream) {
    case [] -> Nothing
    case (x :: xs) -> { do Put((x :: first(s), xs)); k(x) }
} }
\end{lstlisting}
First the handler retrieves the current parsing state, and then pattern matches on the current state of the input stream. In the event that the input stream is empty the handler returns \type{Nothing}. Otherwise, it consumes the next character \code{x} and cons it to the list of parsed symbols. Thereafter the continuation is invoked to return the character \code{x} to the \code{token}-parser that discharged \type{Token}.
If the subsequent parsing is successful, then the \code{k} returns \code{Just} the result, otherwise it returns \code{Nothing}.
The case for \type{Choose} follows a similar pattern:
\begin{lstlisting}[style=links]
case Choose(_,k) -> {
  var s = do Get();
  switch (k(true)) {
    case Nothing -> { do Put(s); k(false) }
    case Just(x) -> Just(x)
} }
\end{lstlisting}
Again the handler retrieves the current parsing state \code{s}. Thereafter we pattern match on the result of picking the \code{true}-branch. If it leads to failure, then we restore the previous state \code{s} by discharging a \type{Put}-operation, and then subsequently try the \code{false}-branch. Further, \code{k} returns either \code{Just} the result or \code{Nothing}. If the choice led to success, then we simply return the identity.
Finally, we implement the \type{Return}-case which is somewhat similar to the \type{Token}-case, e.g.
\begin{lstlisting}[style=links]
case Return(x)  -> {
  var s = do Get();
  var stream = second(s);
  switch(stream) {
    case []    -> Just(reverse(first(s)))
    case other -> Nothing
} }
\end{lstlisting}
Here, we pattern match on the input stream to determine whether all input has been consumed. If the stream is empty, then we return \code{Just} the reversed list of parsed symbols, otherwise we return \code{Nothing}. As a final function we implement a convenient driver function \code{parse} to abstract away the details of running the \code{parser}-handler. The \code{parse} function takes a parser \code{p} and a string \code{source} as input parameters:
\begin{lstlisting}[style=links]
fun parse(p, source) {
  var s = ([], explode(source));
  switch (runState(parser(p), s)) {
     case Just(r) -> Just(implode(r))
     case Nothing -> Nothing
} }
\end{lstlisting}
The function builds the initial state \code{s}, where the first component is the empty list, and the second component contains the input string as a list of characters. The \code{switch}-expression runs the parser with the initial state, and pattern matches on the result. If the parser was successful, then the list of characters is converted back into a string, otherwise it returns the identity.
\subsubsection{Parsing palindromes}
The previous two sections provided the basic building blocks for parsing. Now, we are ready to put them into action. Example \ref{ex:palindromes} demonstrates how to parse the {\scshape{Palindromes}} language.
\begin{example}[{\scshape{Palindromes}} parser]\label{ex:palindromes}
We implement a parser for the grammar \eqref{eq:grammar} using parser combinators. The structure of the resulting parser will closely resemble the structure of the grammar e.g.
\begin{lstlisting}[style=links]
fun p() {
  var a   = char('a'); var b = char('b');
  var apa = a <*> p <*> a;
  var bpb = b <*> p <*> b;
  var p = apa <|> bpb <|> a <|> b <|> empty;
  p()
}
\end{lstlisting}
The function \code{p} is a parser as its type is $() \xrightarrow{E} ()$. The first line in \code{p} constructs two parsers \code{a} and \code{b} which parses a single character each. Next, the parser \code{apa} parses a palindrome which starts with an 'a'. Similarly, \code{bpb} parses a palindrome starting with a 'b'.
Finally, the parsers are combined to form a parser for the {\scshape{Palindromes}} grammar \eqref{eq:grammar}.
Notably, the definition of the resulting parser \code{p} corresponds closely to the definition of $P$ in grammar \eqref{eq:grammar}.
Running the parser on a few examples we obtain:
\begin{lstlisting}[style=links]
links> parse(p, "abba");
Just("abba") : Maybe(String)
links> parse(p, "bbbbaabaabbbb");
Just("bbbbaabaabbbb") : Maybe(String)
links> parse(p, "");
Just("") : Maybe(String) # The empty palindrome
links> parse(p, "aaabbb");
Nothing() : Maybe(String)
\end{lstlisting}
\end{example}

The entire implementation is less than 100 lines. Yet, the framework is quite general. Shifting all parsing state maintenance to a handler greatly simplifies parser combinators. 