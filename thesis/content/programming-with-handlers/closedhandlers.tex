\section{Closed handlers}\label{sec:closedhandlers}
A handler assigns semantics to abstract operations. In Links, this is reflected in the syntax as a handler embodies a collection of pattern-matching cases, which map operation names to computations, e.g.
\begin{lstlisting}[style=links]
handler h(m) {
  case Get(p,k)  -> §*$m_{\code{Get}}$*§
  case Put(p,k)  -> §*$m_{\code{Put}}$*§
  case Return(x) -> §*$m_{\code{Return}}$*§
}
\end{lstlisting}
The above defines a handler \code{h} that handles two operations \code{Get} and \code{Put}.
In the previous section we said that every operation takes exactly one argument, yet, an operation case matches on two parameters. The first parameter \code{p} is the operation argument, whilst the second parameter \code{k} exposes a delimited continuation that accepts a single parameter. Invocation of the continuation transfers control back to the handled computation \code{m} at the point where the said operation was discharged. Both parameters may be referenced multiple times in their respective case-computation $m_{\code{Get}}$ or $m_{\code{Put}}$.
There may be a variable number of operation-cases, however, there must be at least one \code{Return}-case in every handler. The \code{Return}-case is a special case that is implicitly invoked when the handled computation \code{m} finishes. The purpose of \code{Return} is discussed in Section \ref{sec:transform}.

The formal parameter \code{m} is a name for the abstract computation which the handler interprets. Because Links employ a strict evaluation strategy computations are modelled as thunks, that is, the type of \code{m} is $() \xrightarrow{E} b$ where $E$ is the set of operations that \code{m} may discharge.

\subsection{Typing closed handlers}
A closed handler handles a fixed set of effects, that is, it puts an upper bound on which kind of effects a computation may perform. In Links this bound is made explicit in the handler's type, e.g. the closed handler \code{h} above has the following type
\[ (() \xrightarrow{ \{\type{Get}: a_1 \to a_2, \type{Put}: a_3 \to a_4 \}\;} b) \to c \]
where $b$ is the return type of the computation \code{m}, and $c$ is the type of $m_{\code{Return}}$, $m_{\code{Get}}$ and $m_{\code{Put}}$.
The absence of a row variable in the effect signature implies that the computation \code{m} may not perform any other effects than \type{Get} and \type{Put}. It is considered a type error to attempt to handle a computation whose effect signature is not unifiable with the effect signature of \code{m}.

This restriction introduces slack into the type system \cite{Huttel2010}. To illustrate the slack consider the following computation:
\begin{lstlisting}[style=links]
fun comp() {
  s = do Get();
  if (false == true) { do Foo(); s }
  else { true }
}
\end{lstlisting}
The computation \code{comp} has type $() \xrightarrow{ \{ \type{Get}:() \to \type{Bool},\type{Foo}:() \to () \; | \; \rho \}\;} \type{Bool}$. Obviously, \code{Foo} never gets discharged. However, attempting to handle \code{comp} with the handler \code{h} gives rise to the following unsolvable equation 
\[ \{\type{Get}: a_1 \to a_2, \type{Put} : a_3 \to a_4 \} \sim \{\type{Get}:() \to \type{Bool},\type{Foo}:() \to () \; | \; \rho \}\]
There is no solution as we cannot remove \code{Foo} from the right hand side.

The following sections will show increasingly interesting examples of programming with closed handlers in Links.

\subsection{Transforming the results of computations}\label{sec:transform}
The first two examples show how to transform the output of a computation using handlers. We begin with a handler that appears to be rather boring, but in fact proves very useful as we shall see later in Section \ref{sec:openhandlers}.
\begin{example}[The force handler]\label{ex:force}
We dub the handler \code{force} as it takes a computation (thunk) as input, evaluates it and returns its result. It has type $(() \to a) \to a$ and it is defined as
\begin{lstlisting}[style=links]
handler force(m) {
    case Return(x) -> x
}
\end{lstlisting}
Essentially, this handler applies the identity transformation to the result of the computation \code{m}. Running \code{force} on a few examples should yield no surprises:
\begin{lstlisting}[style=links]
fun fortytwo() { 42 }
links> force(fortytwo);
42 : Int

fun hello() { "Hello" }
links> force(hello);
"Hello" : String
\end{lstlisting}
The handler \code{force} behaves as expected for these trivial examples. The \code{force} handler also runs side-effecting computations:
\begin{lstlisting}[style=links]
links> force(fun() { print("Hello World") });
"Hello World"
() : ()
\end{lstlisting}
The \code{force} handler's effect row implicitly contains the \code{wild} effect. Without the presence of the \code{wild} effect closed handlers would not be able to run computations that might diverge. Therefore disallowing the \code{wild} effect severely limit the class of computations that a closed handler can accept as input.
\end{example}

The next example demonstrates an actual transformation.
\begin{example}[The listify handler]\label{ex:listify}
The \code{listify} handler transforms the result of a handled computation into a singleton list. Its type is $ (() \to a) \to [a]$ and its definition is straightforward:
\begin{lstlisting}[style=links]
handler listify(m) {
  case Return(x) -> [x]
}
\end{lstlisting}
Running it on a few examples we obtain:
\begin{lstlisting}[style=links]
links> listify(fortytwo);
[42] : [Int]
links> listify(hello);
["Hello"] : [String]
links> listify(fun() { [1,2,3] });
[[1,2,3]] : [[Int]]
\end{lstlisting}
This example also illustrates that the \code{Return}-case serves a similar purpose to the monadic \code{return}-function in Haskell which lifts a value into a monadic value, similarly, the \code{Return}-case lifts a value into a ``handled'' value.
\end{example}

In a similar fashion to the handler \code{listify} in Example \ref{ex:listify} we can define handlers that increment results by 1, perform a complex calculation using the result of the computation, or wholly ignore the result. However, it must ensure that the type of the output is compatible with the output type of the handler. In the case for \code{listify} the output must be a list of whatever the computation yielded.

\subsection{Exception handling}\label{sec:maybehandler}
Until now we have only seen some simple transformations. Let us make things more interesting.
Example \ref{ex:maybe} introduces our first practical handler \code{maybe}. It is similar to the \type{Maybe}-monad in Haskell. For reference we briefly sketched the behaviour of the \type{Maybe}-monad in Section \ref{sec:problem-with-monads}.
\begin{example}[The maybe handler]\label{ex:maybe}
The \code{maybe} handler handles one operation $\type{Fail} : a_1 \to a_2$ that can be used to indicate that something unexpected has occurred in a computation. The handler returns \code{Nothing} when \code{Fail} is raised, and \code{Just} the result when the computation succeeds, thus its type is
\[ \chntype{\thunktype{\type{Fail} : a_1 \to a_2}{b}}{\type{Maybe}(b)}. \]
It is defined as
\begin{lstlisting}[style=links]
handler maybe(m) {
  case Fail(_,_) -> Nothing
  case Return(x) -> Just(x)
}
\end{lstlisting}
When a computation discharges \type{Fail} the handler discards the remainder of the computation and returns \type{Nothing} immediately, e.g.
\begin{lstlisting}[style=links]
fun yikes() { 
  var x = "Yikes!";
  do Fail();
  x
}
links> maybe(yikes);
Nothing() : Maybe(String)
\end{lstlisting}
and if the computation succeeds it wraps the result inside a \code{Just}, e.g.
\begin{lstlisting}[style=links]
fun success() {
  true
}
links> maybe(success);
Just(true) : Maybe(Bool)
\end{lstlisting}
\end{example}
% The next example demonstrates an alternative ``exception handling strategy''.
% \begin{example}[The recover handler]\label{ex:recover}
% We can define a handler \code{recover} which ignores the raised exception and resumes execution of the computation.
% The type of \code{recover} is
% \[ \code{recover} : \chntype{\thunktype{\type{Fail}: a \to ()}{b}}{[|\type{Just}:b|\rho|]}. \]
% A slight reminder here: The label \type{Nothing} is absent from the handler's output type because \type{Just} is a polymorphic variant label and its relation to \type{Nothing} is conventional. We define \code{recover} as
% \begin{lstlisting}[style=links]
% var recover = handler(m) {
%   case Fail(_,k) -> k(())
%   case Return(x) -> Just(x)
% }
% \end{lstlisting}
% In contrast to \code{maybe} from Example \ref{ex:maybe} the \code{recover} handler invokes the continuation \code{k} once. This invocation effectively resumes execution of the computation. Consider \code{recover} applied to the computation \code{yikes} from before
% \begin{lstlisting}[style=links]
% links> recover(yikes)
% Just("Yikes!") : [|Just:String|§*$\rho$*§|]
% \end{lstlisting}
% \end{example}
% Although it is seldom a sound strategy to ignore exceptions the two Examples \ref{ex:maybe} and \ref{ex:recover} demonstrate that we can change the semantics of the computation by changing the handler.

\subsection{Handling choice}\label{sec:choice}
In Section \ref{sec:effects-as-computation-trees} we visualised some interpretations the abstract computation
\begin{lstlisting}[style=links]
fun choice() {
  var x = if (do Choose()) { 2 }
          else { 4 };
  var y = if (do Choose()) { 8 }
          else { 16 };
  y + x
}
\end{lstlisting}
which uses the operation $\type{Choice} : () \to \type{Bool}$. For completeness we show how to implement these interpretations in Links. Example \ref{ex:positive} shows the positive interpretation and Example \ref{ex:enumerate} shows the enumeration interpretation.
\begin{example}[The ``positive'' interpretation (Figure \ref{fig:positive})]\label{ex:positive}
Whenever the operation \type{Choose} is discharged in the computation \code{choice} the handler has to decide whether to pick \code{true} or \code{false}. Therefore the type of the handler \code{positive} must be
$(() \xrightarrow{\{\type{Choose}:() \to \type{Bool}\}\;} a) \to a$.
The \code{positive} handler always picks \code{true}. To implement this behaviour, we invoke the continuation once with the argument \code{true}. The value \code{true} becomes the concrete output of \code{do Choose} in the computation:
\begin{lstlisting}[style=links]
handler positive(m) {
  case Choose(_,k) -> k(true)
  case Return(x)   -> x
}
\end{lstlisting}
Running the handler on the computation \code{choice} yields the expected result:
\begin{lstlisting}[style=links]
links> positive(choice);
10 : Int
\end{lstlisting}
The definition of the handler \code{negative} from Section \ref{sec:effects-as-computation-trees} is analogous to \code{positive}.
\end{example}

\begin{example}[The enumeration handler (Figure \ref{fig:enumerate})]\label{ex:enumerate}
The handler \code{enumerate} traverses the entire computation tree as shown in Figure \ref{fig:enumerate}. To encode this behaviour we will invoke the continuation twice: First with \code{true} and then with \code{false}. The results of both invocations have to be collected in a list. It is the job of \code{Return} to lift a single result into a list. Therefore the type of \code{enumerate} is 
$(() \xrightarrow{\{\type{Choose}:() \to \type{Bool}\}\;} a) \to [a]$.
The \code{Return}-case lifts a single element into a singleton list. Hence the two invocations of the continuation give us two lists which we can join together to form a single list, e.g.
\begin{lstlisting}[style=links]
handler enumerate(m) {
  case Choose(_,k) -> k(true) ++ k(false)
  case Return(x)   -> [x]
}
\end{lstlisting}
Applying \code{enumerate} to the computation \code{choice} yields the result we arrived at in Section \ref{sec:effects-as-computation-trees}:
\begin{lstlisting}[style=links]
links> enumerate(choice);
[10, 18, 12, 20] : [Int]
\end{lstlisting}
\end{example}

\subsection{Interpreting Nim}\label{sec:interpreting-nim}
The previous examples built intuition for how handlers work. In this section we will use the mathematical game Nim to demonstrate the power of modularity afforded by handlers. Nim is a strategic game in which two players take turns to pick sticks from heaps on a table, and whoever takes the last stick wins.
We will use a simplified version of Nim to demonstrate how handlers can give different interpretations of the same game. In our simplified version there is only one heap of $n$ sticks. Any player may only take between one and three sticks at a time. Furthermore, we name players: Alice and Bob, and Alice always starts. Our model is adapted from Kammar et. al \cite{Kammar2013}.

We encode the players as the sum type $\type{Player} \defas [|\code{Alice}|\code{Bob}|]$ and model the game as two mutual recursive abstract computations, e.g.
\begin{figure*}[h]
  \centering
  \begin{tabular}{ c | c }
When Alice plays  & When Bob plays \\
\hline
    \begin{subfigure}[c]{0.51\linewidth}
\begin{lstlisting}[style=links]
fun aliceTurn(n) {
  if (n == 0) { Bob }
  else {
   var take = do Move((Alice,n));
   var r = n - take;
   bobTurn(r)
} }
\end{lstlisting}
    \end{subfigure}
     &
    \begin{subfigure}[c]{0.49\linewidth}
\begin{lstlisting}[style=links]
fun bobTurn(n) {
  if (n == 0) { Alice }
  else {
   var take = do Move((Bob,n));
   var r = n - take;
   aliceTurn(r)
} }
\end{lstlisting}
    \end{subfigure}
\end{tabular}
\end{figure*}

The two computations are symmetrical. The input parameter \code{n} is the number of sticks left in the heap. First, Alice checks whether there are any sticks left, if the heap is empty then she declares \type{Bob} the winner, otherwise she performs her move and then she gives the turn to Bob.
The game uses one abstract operation \code{Move} which has the type $\code{Move} : (\type{Player}, \type{Int}) \to \type{Int}$, i.e. it takes the current game configuration as input:
\begin{enumerate}
  \item Who's turn it is,
  \item and the number of remaining sticks.
\end{enumerate} 
The operation \code{Move} returns the number of sticks that the current player takes. At this point \code{Move} does not have a clear semantic interpretation. We only know that its range, the integers, is infinite, so $\code{take}$ may be assigned any integer value.
The types of \code{aliceTurn} and \code{bobTurn} are
\[ (\type{Int}) \xrightarrow{\{\type{Move}:(\type{Player},\type{Int}) \to \type{Int} \; | \; \rho\}\;} \type{Player} \]
Because it is an unary function it cannot be used as an input to any handler. We rectify the problem by using a closure, i.e. we wrap the game function inside a nullary function like \code{fun()$\{$aliceTurn(n)$\}$} where \code{n} is a free variable captured by the surrounding context.
For conveniency, we define an auxiliary function \code{play} to abstract away these details. It takes as input a game handler \code{gh} and the number of sticks at the beginning of game \code{n}. Moreover, the function \code{play} enforces the rule that Alice always starts, e.g.
\begin{lstlisting}[style=links]
fun play(gh, n) {
  gh(fun() {
      aliceTurn(n)
    })
}
\end{lstlisting}
The following examples demonstrates how to use handlers to encode the strategic behaviour of the players.
\begin{example}[A naïve strategy]\label{ex:nim-naive}
A very naïve strategy is to take only \emph{one} stick at every turn. We encode this behaviour by invoking the continuation with argument $1$. This assigns $1$ to \code{take} in \code{aliceTurn} or \code{bobTurn} depending on whom discharged \code{Move}.
The implementation of the handler is straightforward:
\begin{lstlisting}[style=links]
handler naive(m) {
  case Move(_,k) -> k(1)
  case Return(x) -> x
}
\end{lstlisting}
The operation \code{Move} is handled uniformly, that is, independent of the current game configuration the handler always invokes the continuation \code{k} with 1.

A moment's thought reveals that we can easily predict the winner when using the \code{naive} strategy. The parity of $n$, the number of sticks at the beginning, determines the winner. For odd $n$ Alice wins and vice versa for even $n$, e.g.
\begin{lstlisting}[style=links]
links> play(naive, 9);
Alice() : Player

links> play(naive, 18);
Bob() : Player

links> play(naive, 101);
Alice() : Player
\end{lstlisting} 
\end{example}

\begin{example}[A perfect strategy \cite{Kammar2013}]\label{ex:nim-perfect}
A perfect strategy makes an optimal move at each turn. In particular, an optimal move depends on the remaining number of sticks $n$. Therefore the perfect move can be defined as a function of $n$, e.g.
\[ \text{perfect}(n) = \max\{n\; \text{mod}\; 4, 1\} \]
In our restricted Nim game a perfect strategy is a winning strategy for Alice if and only if the number of remaining sticks is \emph{not} divisible by 4.

We implement the function \code{perfect} above with an addition: We pass it a continuation as second parameter:
\begin{lstlisting}[style=links]
fun perfect(n, k) {
  k(max(mod(n,4),1))
}
\end{lstlisting}
The continuation is invoked with the optimal move. Now we can easily give a handler that assigns perfect strategies to both Alice and Bob, e.g.
\begin{lstlisting}[style=links]
handler pvp(m) {
  case Move((_,n),k) -> perfect(n, k)
  case Return(x)     -> x
}
\end{lstlisting}
Notice that this time we pattern match on \code{Move}'s argument to obtain the game configuration.
By running some examples we witness that Alice wins when $n$ is not divisible by four:
\begin{lstlisting}[style=links]
links> play(pvp, 9);
Alice() : Player

links> play(pvp, 18);
Alice() : Player

links> play(pvp, 36);
Bob() : Player
\end{lstlisting}
\end{example}

\begin{example}[Mixed strategies]\label{ex:nim-mixing}
A strategy often encountered in game theory is \emph{mixing} which implies a player randomises its strategies in order to confuse the opponent. In similar fashion to \code{perfect} from Example \ref{ex:nim-perfect} we define a function \code{mix} which chooses a strategy
\begin{lstlisting}[style=links]
fun mix(n,k) {
  var r = mod(nextInt(), 3) + 1;
  if (r > 1 && n >= r)) { k(r) }
  else { k(1) }
}
\end{lstlisting}
The function \code{nextInt} returns the next integer in some random sequence. The random integer is projected into the set of valid moves $\{1,2,3\}$. If the random choice \code{r} is greater than the number of remaining sticks \code{n} then we default to take one (even though the optimal choice might be to take two).

The \code{mixed} strategy handler is similar to perfect-vs-perfect handler from Example \ref{ex:nim-perfect}
\begin{lstlisting}[style=links]
handler mixed(m) {
    case Move((_,n),k) -> mix(n,k)
    case Return(x)     -> x			  
}
\end{lstlisting}
Replaying the same game a few times ought eventually yield the two possible outcomes:
\begin{lstlisting}[style=links]
links> play(mixed, 9);
Bob() : Player

links> play(mixed, 9);
Alice() : Player
\end{lstlisting}
\end{example}

\begin{example}[Brute force strategy \cite{Kammar2013}]\label{ex:nim-brute-force}
Examples \ref{ex:nim-naive}-\ref{ex:nim-mixing} only invoked the continuation once per move. However, we can invoke the continuation multiple times to enumerate all possible future moves, this way we can brute force a winning strategy, if one exists. In order to brute force a winning strategy, we define a convenient utility function which computes the set of valid moves given the number of remaining sticks:
\begin{lstlisting}[style=links]
fun validMoves(n) {
  filter(fun(m) { m <= n }, [1,2,3])
}
\end{lstlisting}
The function simply filters out all illegal moves for a given game configuration $n$.
The function \code{bruteForce} computes the winning strategy for a particular player if such a strategy exists:
\begin{lstlisting}[style=links]
fun bruteForce(player, n, k) {
  var winners = map(k, validMoves(n));
  var hasPlayerWon = indexOf(player, winners);
  switch (hasPlayerWon) {
     case Nothing -> k(1)
     case Just(i) -> k(i+1)
} }
\end{lstlisting}
The first line inside \code{bruteForce} is the critical point. Here we map the continuation \code{k} over the possible moves in the current game configuration. Accordingly, the function simulates all possible future configurations yielding a list of possible winners. The auxiliary function \code{indexOf} looks up the position of \code{player} in the list of winners. The position plus one corresponds to the winning strategy because lists indexes are zero-based. If the player has a winning strategy then the (zero-based) position is returned inside a \type{Just}, otherwise \code{Nothing} is returned.

We let Alice play the brute force strategy and Bob play the perfect strategy:
\begin{lstlisting}[style=links]
handler bfvp(m) {
  case Move((Alice,n),k) -> bruteForce(Alice,n,k)
  case Move((Bob,n),k)   -> perfect(n,k)
  case Return(x)         -> x
}
\end{lstlisting}
Here we take advantage of deep pattern-matching to distinguish between when \type{Alice} and \type{Bob}'s moves. Obviously, the brute force strategy is inefficient as it redoes a lot of work for each move. The winning strategy that it discovers happens to be the same as the perfect strategy. Albeit, \code{bruteForce} computes it in exponential time whilst \code{perfect} computes it in constant time. The following outcomes witness that \code{bruteForce} and \code{perfect} behaves identically:
\begin{lstlisting}[style=links]
links> play(bfvp, 9);
Alice() : Player

links> play(bfvp, 18);
Alice() : Player

links> play(bfvp, 36);
Bob() : Player
\end{lstlisting}
\end{example}
Although, the \code{bruteForce} strategy is significantly slower than \code{perfect} strategy in Example \ref{ex:nim-brute-force} the point of interest here is not efficiency but rather modularity. Remark that during Examples \ref{ex:nim-naive}-\ref{ex:nim-brute-force} the game model remained unchanged. We interpreted the game by instantiating the operation \code{Move} with different implementations. 
Moreover, Example \ref{ex:nim-brute-force} nicely demonstrated that we may exchange two observable equivalent implementations (handlers) effortlessly. In practical terms this implies that one would be able to exchange a slow component with a faster, improved version effortlessly. The coupling between the game model and the handlers is low as they interface through the abstract operation \code{Move}. Furthermore, the handlers followed a similar pattern. It would be convenient to be able to abstract over this pattern by defining a generic game handler that, in addition to an abstract computation, takes two strategies as input. However, in the current implementation handlers cannot be parameterised.

Examples \ref{ex:nim-naive}-\ref{ex:nim-brute-force} gave different interpretations of the same game. Furthermore, they all computed the same thing, namely, the winner. In particular, each handler applied the identity transformation in the \code{Return}-case. However, by taking full advantage of the \code{Return}-case we can use handlers to compute data from computations. For example we can construct the game tree for a Nim game as Example \ref{ex:nim-game-tree} shows. 

\begin{example}[Game tree generator \cite{Kammar2013}]\label{ex:nim-game-tree}
In a game tree a node represents a particular player's turn, and outgoing edges corresponds to particular moves that the player may perform. A path down the game tree corresponds to a particular sequence of moves taken by the players ending in a leaf node which corresponds to the winner. Figure \ref{fig:ex-nim-game-tree} illustrates an example game tree when starting with 3 sticks. Our game tree is a ternary tree which we represent using a recursive variant type, e.g.
\[ \type{GameTree} \defas [|\type{Take}:(\type{Player},[(\type{Int},\type{GameTree})])\,|\,\type{Winner}:\type{Player}|] \]
Leaves are tagged with \code{Winner}, while nodes are tagged with \code{Take}. A \code{Take} node embeds a tuple where the first component is current player, and the second component contains the possible subgames.
We define a function \code{reifyMove} which takes a player, the number of sticks, and a continuation to construct a node in the game tree, e.g.
\begin{lstlisting}[style=links]
fun reifyMove(player, n, k) {
  var moves = map(k, validMoves(n));
  var subgames = zip([1..length(moves)], moves);
  Take(player, subgames)
}
\end{lstlisting}
First, we map the continuation over the possible moves in the current game configuration to enumerate the subsequent game trees. We compute the subgames by zipping element-wise the two list $\{1,\dots,|\code{moves}|\}$ and \code{moves}. Finally, we construct a node \type{Take} with \code{player} and the possible subsequent game trees.

The leaves are constructed by the \code{Return}-case in the handler:
\begin{lstlisting}[style=links]
handler gtGen(m) {
  case Move((player,n),k) -> reifyMove(player,n,k)
  case Return(x)          -> Winner(x)
}
\end{lstlisting}
Again, we take advantage of full pattern-matching to decompose the game configuration. The inferred type for \code{gtGen} witnesses that the handler indeed constructs a game tree:
\[ \code{gtGen} : (() \xrightarrow{\{\type{Move}:(\type{Player},\type{Int}) \to \type{Int}\}\;} \type{Player}) \to \type{GameTree} \]
Figure \ref{fig:ex-nim-game-tree} depicts the game tree generated by the handler when starting with 3 sticks.
\begin{figure}[H]
\begin{center}
\begin{tikzpicture}[level distance=1.5cm,
level 1/.style={sibling distance=3.5cm},
level 2/.style={sibling distance=2cm}]

\node (Root) [draw=none,rectangle] {Alice}
    child { node[draw=none] (q0) {Bob} 
      child { node[draw=none] {Alice}
        child { node[draw=none] {Alice wins}
          edge from parent node [draw=none,left,xshift=-2.3,yshift=2.3] {1}
        }
        edge from parent node [draw=none,left,xshift=-2.0,yshift=2.0] {1}
      }
      child { node[draw=none] {Bob wins}
        edge from parent node [draw=none,right,xshift=2.0,yshift=2.0] {2}
      }
      edge from parent node [draw=none,left,xshift=-2.0,yshift=2.0] {1}
    }
    child { node [draw=none] {Bob}
      child { node[draw=none] {Bob wins}
        edge from parent node [draw=none,right,xshift=2.0,yshift=2.0] {1}
      }
      edge from parent node [draw=none,right,xshift=2.0,yshift=2.0] {2}
    }
    child { node [draw=none] {Alice wins}
      edge from parent node [draw=none,right,xshift=2.0,yshift=2.0] {3}
    };    
\end{tikzpicture}
\end{center}
\caption{Pretty print of the game tree generated by \code{play(gtGen, 3)}.}\label{fig:ex-nim-game-tree}
\end{figure}
\end{example}
Notice that even though the game model remains unchanged we have been able to encode strategic behaviours and generate data from the game by interpreting the game using different handlers. This emphasises the modularity afforded by handlers.