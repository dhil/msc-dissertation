\section{Related work}\label{sec:relatedwork}
This section discusses and evaluates related work on programming models with handlers and effects.

\subsection{Languages with effects}
\paragraph{Eff.} The \emph{Eff} programming language, by Bauer and Pretnar \cite{Bauer2015}, has first-class support for effect handlers and algebraic effects. The language has the look and feel of OCaml. Eff employs nominal typing for effects, therefore an effect is a named collection of abstract operations. To discharge an operation, the programmer has to generate an effect instance. Operations can be discharged through the effect instance. This interface is similar to the object-oriented interface in OCaml. 
Eff achieves unordered effects through effect subtyping. 

\paragraph{Frank.} The Frank programming language by McBride \cite{McBride2014} takes the notion of effect handlers to the extreme. In Frank there are no functions, there are only handlers. Consequently, a function is a special case of a handler.
In particular, it only supports \emph{shallow} handlers whereas our implementation only supports \emph{deep} handlers. Shallow handlers handle computations nonuniformly. Additionally, Frank employs ``call-by-push-value'' (CBPV) evaluation semantics. Essentially, CBPV makes the distinction between computations and values explicit. Since Frank distinguishes between computations and values side-effects can only occur in computations. Hence there is a clear separation between segments of code where effects might occur and where effects are guaranteed never to occur.

\paragraph{Koka with row polymorphic effects.}
Leijen's programming language Koka employs a row-based effect system \cite{Leijen2014} with support for arbitrary user-defined effects \cite{Vazou2015}.
Notably, Koka uses row polymorphism to capture unordered effects, however, in contrast to our approach Koka allows an effect to occur multiple times in a row. Effect duplication is crucial to support scoping of effects.
Furthermore, Koka has no notion of effect handler except for exception handlers which are to some extent reminiscent of those in Java, C\#, etc.

\subsection{Haskell libraries}
There is no first-class support for handlers in Haskell, however there are several embeddings of effect handlers in Haskell. Furthermore, the implementations take advantage of Haskell's lazy evaluation strategy.
%We will discuss two implementations of handlers on top of Haskell by Kammar et. al \cite{Kammar2013} and Wu et. al \cite{Wu2014}.

\paragraph{Data types á la carte.}
In his functional pearl, Swierstra illustrates how to write effectful programs using \emph{free monads} \cite{Swierstra2008}. Free monads gives rise to a natural encoding of effect handlers \cite{McBride2014}.

\paragraph{Handlers in action.}
Kammar et al. \cite{Kammar2013} considers two different implementations of handlers in Haskell. One implementation is based on free monads, while the other is based on continuations.

They achieve unordered effects by encoding handlers as type classes. Therefore handlers also inherit the limitations of type classes. Type classes are not first-class in Haskell, so neither are handlers. Haskell do not permit local type class definitions, therefore handlers must be defined in the top-level.
Furthermore, the order in which handlers are composed leak into the type signature, because their (open) handlers explicitly mention a parent handler \cite{Kammar2013}. However, this is less problematic than the effect ordering problem. They propose that using row polymorphism may yield a cleaner design \cite{Kammar2013}. To a large extent our implementation is inspired by their work. Moreover, they present a collection of examples, some of which we have reproduced in Links. We also take a similar approach to evaluating the relative performance of handlers. 

Notably, they provide a Template Haskell interface which makes it easier to use their library.

\paragraph{Extensible effects.}
Kiselyov et al. presents an alternative framework to Monad Transformers based on free monads \cite{Kiselyov2013}. Their framework is modelled after the syntax of Monad Transformers, but, allows effects to be combined in any order. Like Frank their framework only supports shallow handlers which gives the programmer additional flexibility, but, shallow handlers are often less efficient than deep handlers \cite{Kiselyov2013}.

\paragraph{Handlers in scope}
Wu et al. investigates scoping constructs for handlers to delimit the scope of effects \cite{Wu2014}.

They present two solutions: The first solution extends an existing effect handler framework based on free monads with so-called \emph{scope markers} to mark the beginning and ending of blocks that should be handled in a self-contained context. However they demonstrate that handlers along with scope markers are insufficient to capture higher-order scoped constructs properly.
Their second approach provides a \emph{higher-order syntax} which carries the scoped blocks directly \cite{Wu2014}. In addition, the second approach gives finer control how handlers traverse the syntax tree. %However it remains an open question whether their implementation is viable in other languages than Haskell.

\subsection{Idris' Effects}
Brady provide a library implementation of handlers called {\scshape{Effects}} for the dependently-typed, functional programming language {\scshape{Idris}} \cite{Brady2013}. The purpose is to investigate the use of effects to reason about programs. So, the type checker does not infer effect types, rather it checks that effects are used correctly according to some specification given by the programmer.