\section{Row polymorphism}\label{sec:rowpolymorphism}
Row polymorphism is a typing discipline for row-based types such as records and variants \cite{Remy1993}. A row is an unordered collection of labels, e.g. 
\[ \type{Employee} \defas \record{\code{name}:\type{String},\code{dept}:\type{String}} \]
denotes a row type with two fields: name of type \type{String} and dept of type \type{String}. We use set notation to emphasise that a particular field may only appear once in a row. The record $\code{emp} \defas \record{\code{name}="John", \code{dept}="Finance"}$ is a possible instance of the row type \type{Employee}. 
%The above row is \emph{closed}, alternatively, if a row contains a row variable $\rho$, then it is said to be \emph{open}, e.g.
%\[
%\type{supervisor} \defas \record{\code{name}:\type{String} \; | \; \rho} 
%\]
%denotes a polymorphic row. Observe that the two above types both contains the label \code{name}. 

Consider a function which returns the projection of the \code{name} component and the row itself, e.g. $\code{get\_name}(r) \defas (\pi_{\code{name}}(r), r)$, where $\pi_l(r)$ is the projection of label $l$ in row $r$. This raises the question of how to type the function. One possible typing is 
\[ \code{get\_name} : \record{\code{name}:\type{a}} \to \type{a} \times \record{\code{name}:\type{a}} \]
The type looks innocuous, however, assuming for a moment that the type \type{Employee} is a subtype of $\record{\code{name}:\type{String}}$, then 
\[
  \code{get\_name}(\code{emp}) = ("John", \record{\code{name}="John"})
\]
The output lost the \code{dept} field, because subtyping subsumed the field. However using row polymorphism, we can prevent the loss of information.
The principal idea is to extend row types with a row variable $\rho$ which can be instantiated with additional fields, thus we may type $\code{get\_name}$ as
\[ \code{get\_name} : \record{\code{name}:\type{a} \; | \; \rho} \to (a \times \record{\code{name}:\type{a} \; | \; \rho}) \]
The row \record{\code{name}:\type{a} \; | \; \rho} is said to be \emph{open} due to the presence of $\rho$, conversely, the row \type{Employee} is said to be \emph{closed}. Additionally, row polymorphism equip field types with a presence flag which indicating whether a field is \emph{present} or \emph{absent} \cite{Remy1993}. We will denote presence by $pre(\tau)$ and absence by $abs$ where $\tau$ is the type of the field in question. If we attempt to apply \code{get\_name} to \type{emp} under this typing then we obtain \code{(``John'', emp)} as desired. Under the hood the type system has to solve the equation 
\[ 
\record{\code{name}:pre(\type{a}) \; | \; \rho} \sim \record{\code{name}:pre(\type{String}),\code{dept}:pre(\type{String})}
\]
that is the two row types must be unified. The goal is to obtain a row where everything that is present on the left and right hand side are present in the solution. %Note that \code{get\_name} reads all fields in the input row $r$, therefore all fields must be present in the solution. 
This implies that rows only can grow monotonically.
The solution is to first instantiate the type variable $a$ with $\type{String}$, and then instantiate $\rho$ with the additional field $\code{dept}$ of type $pre(\type{String})$. The result is a row that is identical in structure to the row type \type{Employee}. It is crucial that the left hand side row is open, otherwise the equation would have no solution.

As a final example, consider the row type $\record{\code{name}:\type{String},\code{at\_job}:\theta}$, where $\theta$ denotes the field is polymorphic in whether it is present or absent. A possible instance is $\record{\code{name}="Paul"}$ because the field \code{at\_job} does not have to be present. Applying the function \code{get\_name} to the row gives rise to a similar equation:
\[ 
\record{\code{name}:pre(\type{a}) \; | \; \rho} \sim \record{\code{name}:pre(\type{String}),\code{at\_job}:abs}
\]
Again, we unify field by field: The case for \code{name} is easy, we simply instantiate the type variable $a$ with \type{String}. Next, we instantiate $\rho$ with \code{at\_job} set to $abs$.  For all practical matter the two row types
\[ 
\record{\code{name}:pre(\type{String})} \approxeq \record{\code{name}:pre(\type{String}),\code{at\_job}:abs}
\] 
are identical in structure, because \code{at\_job} is never accessed in an instance. In the remainder of the thesis we continue to use $\theta$ to denote presence polymorphism, but we omit the presence annotation \emph{pre} as we will mostly work with rows where all fields are present.