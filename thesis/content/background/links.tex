\section{Links toplevel}
Links is a strongly-, statically-typed, web-oriented functional programming language that renders webprogramming tierless by enabling programmers to write their code in one single source language \cite{Cooper2006,Links}. However, we will not address Links' web-oriented features, instead we will give a brief introduction to the functional core language.

\subsection{The basics}
Links is an interpreted language. Launching the Links interpreter leaves us in a read-eval-print-loop (REPL), where we can execute arbitrary Links expressions. We will use the notation ``\code{links>}'' to denote when we use the REPL. The syntax of Links is similar to JavaScript's syntax.

\paragraph{Functions and variables.}
Variables are declared using the \code{var} keyword followed by a variable name and definition:
\begin{lstlisting}[style=links]
links> var x = 42;
x = 42 : Int
\end{lstlisting}
Note, variables are single-assignment, there is no concept of mutable references in Links.
Similarly, functions are declared using the \code{fun} keyword, followed by a list of parameters and body definition. Furthermore, functions can be either anonymous or named:
\begin{lstlisting}[style=links]
links> fun id(x) { x };
id = fun : (a) -> a
links> fun(x) { x };
fun : (a) -> a
\end{lstlisting}
The above functions are the named and unnamed versions of the identity function. A slightly more interesting function is the following:
\begin{lstlisting}[style=links]
fun meaning_of_life(y) {
  if (y == 42) { true }
  else { false } }

links> meaning_of_life(x); # The same x as defined above
true : Bool
links> meaning_of_life(x-1);
false : Bool
\end{lstlisting}
The body of the function \code{meaning\_of\_life} is a conditional expression with two branches. Because it is an \emph{expression} it must return a value, hence a conditional expression must  have at least two branches to cover \code{true} and \code{false} cases.

\paragraph{Type annotation and aliasing.} The Links type checker automatically infers types for expressions. But, we can define our own types using the \code{typename} keyword. Type and data constructors start with a capital letter. For instance, we can define a \type{Maybe} type constructor:
\begin{lstlisting}[style=links]
links> typename Maybe(a) = [|Just:a|Nothing|];
Maybe = a . [|Just:a|Nothing|]
\end{lstlisting}
The variable \code{a} is a type variable. So, the type constructor \code{Maybe} takes one type \code{a} as input. The brackets (\code{[|...|]}) are syntax for denoting variant types. The data constructor \code{Just} wraps an element of type \type{a}. Let us make an instance:
\begin{lstlisting}[style=links]
links> Just(2);
Just(2) : [|Just:Int|§*$\rho$*§|]
\end{lstlisting}
The Links type checker infers type a polymorphic variant type which Links makes it explicit by the presence of $\rho$. Links employ structural typing, that is, two types are equivalent if and only if they have identical structure. Often Links infers a more general type than needed, we can help the type checker by annotating an expression:
\begin{lstlisting}[style=links]
links> Just(2) == (Just(2) : Maybe(Int));
true : Bool
\end{lstlisting}
Indeed, the two types are compatible due to row polymorphism.
% \begin{lstlisting}[style=links]
% links> typename BinaryTree = [|Leaf:Int|Node:(BinaryTree,Int,BinaryTree)|]
% BinaryTree = mu a . [|Leaf:Int|Node:(a, Int, a)|]
% \end{lstlisting}
% The brackets (\code{[|...|]}) are syntax for denoting variant types. The type reads: A binary tree is either a \code{Leaf} with an \code{Int} or a \code{Node} with a triple \code{(a,Int,a)}, where \code{a} is a recursive type variable which Links makes explicit by including \code{mu} in the signature.
% Let us make an instance of \type{BinaryTree}:
% \begin{lstlisting}[style=links]
% links> var tree = Node(Leaf(0),1,Leaf(2));
% tree = Node((Leaf(0), 1, Leaf(2))) : [|Node:([|Leaf:Int|§*$\rho_1$*§|], Int, [|Leaf:Int|§*$\rho_2$*§|])|§*$\rho_3$*§|]
% \end{lstlisting}
% The Links type checker infers type a polymorphic variant type which Links makes it explicit by the presence of $\rho_i$. Links employ structural typing, that is, two types are equivalent if and only if they have identical structure. Often Links infers a more general type than needed, we can help the type checker by annotating an expression:
% \begin{lstlisting}[style=links]
% links> tree == (tree : BinaryTree);
% true : Bool
% \end{lstlisting}
% Indeed, the two types are compatible due to row polymorphism.

\paragraph{Pattern matching.} We can use the \code{switch}-construct to pattern match on an expression. For example, we can define a function converts an instance of \code{Maybe} into an instance of \type{Bool}:
\begin{lstlisting}[style=links]
fun maybe2bool(maybe : Maybe(String)) {
  switch(maybe) {
    case Just(p)     -> true
    case Nothing     -> false
} }
links> maybe2bool(Just(42));
true : Bool
\end{lstlisting}
The \code{case}-statements pattern matches on the expression \code{maybe}. Links supports deep pattern-matching, so the \code{switch}-construct can be used to decompose an expression.

\subsection{The effect system}
Links has a row-based effect system.