\chapter{Introduction}
A recipe for the ideal programming model would include: Compositionality, modularity and explicit effects.

Compositionality lets us break a complex problem into smaller constituent problems. The complexity of a greater problem can be harnessed by composing solutions to smaller, likely easier, constituent problems. Moreover, compositionality encourage reuse of specialised components to solve future problems.

Modularity refers to the degree of coupling between components. A high degree of modularity implies low coupling between components. Low coupling can be achieved by keeping interfaces between connected components abstract. Abstract interfaces lets us exchange one concrete implementation for another implementation effortlessly.

Together modularity and compositionality form the basis for a powerful programming model. However, being explicit about effects is often neglected \cite{Meijer2014}. An effect give a static description of the possible state-changing actions that may occur during evaluation of a particular piece of code. Moreover, effects can be informative for the compiler as well as the programmer \cite{Kammar2012,Meijer2014}.

Plotkin and Pretnar's \emph{handlers for algebraic effects} \cite{Plotkin2013} afford a compelling programming model which unifies the compositionality, modularity and effectful programming. We will examine the programming model as basis for effectful programming.