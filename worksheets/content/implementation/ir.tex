\section{Pattern-matching compilation}
Syntactically, the \code{handler}-construct and \code{switch}-construct are similar. Figure \ref{fig:handler-switch} depicts their similarities.
\begin{figure}[h]
    \centering
    \begin{subfigure}[c]{0.45\textwidth}
        \centering
\begin{lstlisting}[style=links]
handler(m) {
  case Op§*$_i$*§(p§*$_i$*§,k§*$_i$*§) -> b§*$_i$*§
  case Return(x) -> b
}
\end{lstlisting}        
    \end{subfigure}%  
    ~
    \begin{subfigure}[c]{0.45\textwidth}
        \centering
\begin{lstlisting}[style=links]
switch(e) {
  case §*$\textit{Pattern}_j$*§ -> b§*$_j$*§
  case other   -> b§*$'$*§
}
\end{lstlisting}       
    \end{subfigure}
\caption{The \code{handler}-construct resembles the \code{switch}-construct syntactically.}\label{fig:handler-switch}
\end{figure}
Notably, their semantics differ as \code{switch} allows arbitrary pattern matching on an expression $x$ and \code{handler} only allows pattern matching on possible operation names in some computation $m$. Furthermore, \code{switch} has a default case \code{other} which is not allowed in \code{handler}. 
Their syntactic similarities give rise to a similar internal representation as well. Although, the internal representation of \code{handler} contains extra attributes such as whether the handler is open or closed.
The resemblance has certain benefits:
\begin{itemize}
  \item Syntactical commonalities makes handlers feel like a natural integrated part in Links,
  \item and we can reuse the \code{switch} pattern-matching compilation infrastructure for \code{handler}.
\end{itemize}
The \code{switch} pattern-matching compiler supports deep pattern-matching which we want for matching on actual operation parameters, but only a handful of patterns are permitted for matching on continuation parameters. Figure \ref{fig:cont-pattern-matching} shows the legal pattern-matching on a continuation parameter. Moreover, the \code{Return}-case must only take one parameter. These small subtleties prevent us from using the \code{switch} pattern-matching compiler directly.

\begin{figure}[H]
\begin{center}
\begin{lstlisting}[style=links]
         handler(m) {
           case Op§*$_{i_1}$*§(_,k)      -> b§*$_{i_1}$*§  # Name binding
           case Op§*$_{i_2}$*§(_,k as c) -> b§*$_{i_2}$*§  # Aliasing
           case Op§*$_{i_3}$*§(_,_)      -> b§*$_{i_3}$*§  # Wildcarding
           case Return(x)     -> b§*$_{i_4}$*§
         }
\end{lstlisting}        
\end{center}
\caption{Permissible patterns for matching on the continuation parameter.}\label{fig:cont-pattern-matching}
\end{figure}

Instead we embed the \code{switch} pattern-matching compiler along with a preliminary pattern-matching analyser in the \code{handler} pattern-matching compiler. The pattern-matching analyser checks that the patterns are legal, i.e.
\begin{itemize}
  \item An operation-case has at least two parameters, where the last parameter is supposed to be the continuation.
  \item Pattern-matching on a continuation parameter is either name binding, aliasing or wildcarding.
  \item \code{Return}-case(s) only take one parameter.
\end{itemize}
If the pattern-matching analysis is successful then the \code{switch} pattern matching compiler is invoked to generate the code. Otherwise, a compilation error, complaining about illegal patterns, is emitted.