\section{Type checking}
The type checker implements the following typing rule for open handlers \cite{Kammar2013}:
\begin{equation}\label{eq:typing}
\mprset{flushleft}
\inferrule{E_{in} \defas \{\type{Op}_i : A_i \to B_i\}_i \uplus \rho \\\\
           E_{out} \defas E_{forward} \uplus \rho\\\\
           H \defas \{\code{Return}(x) \mapsto M\} \uplus \{\type{Op}_i(p,k) \mapsto N_i\}_i \\\\
          \left( \varGamma, p : A_i, k : U_{E_{out}}(B_i \to C) \vdash_{E_{out}} N_i : C \right)_i \\\\
          \varGamma, x : A \vdash_{E_{out}} M : C}
          {\varGamma \vdash H : A\, \overset{E_{in}\;\;\;E_{out}}{\Rightarrow} C}
\end{equation}
The typing rule for closed handlers is similar, however, it leaves out the row variable $\rho$.

\subsection{Implementation details}
The type checker for handlers take advantage of the existing infrastructure for the \code{switch}-construct which also embodies a collection of \code{case}-expressions. Figure \ref{fig:handler-switch} displays the two constructs side-by-side.

In order to determine which operations a handler handles the type checker invokes the type checking procedure for \code{case}-expressions. This procedure returns a list of the patterns being matched. In the concrete case for handlers the procedure infers that the \code{case}-expressions pattern match on a variant type. The tags in the variant are precisely the names of the operations that the handler handles. This also reveals why operations resemble variant constructors so closely.

Internally, a variant is represented by a row. So, the handler type checker extracts the row from the inferred variant type, thereafter it applies the typing rule \eqref{eq:typing} to turn obtain the desired effect row.