\begin{frame}
  \frametitle{What this talk is about}
  \begin{center}
    \emph{Handlers for algebraic effects provide a compelling alternative to monads as a basis for effectful programming.}
  \end{center}
\begin{itemize}
  \item \textbf{Key idea:} Separate effect signatures from their implementation.
  \item \textbf{``The effect'':} High-degree of modularity.
\end{itemize}
{\footnotesize{Definitions will follow later\dots}}
\end{frame}

\section{PART 1: Effectively, it's a problem}

\begin{frame}
  \frametitle{Programs are inherently effectful}
  Programs may\dots
  \begin{itemize}
    \item \dots halt prematurely
    \item \dots diverge
    \item \dots be stateful (e.g. modify a global state)
    \item \dots communicate via a network
    \item \dots print to standard out
  \end{itemize}
  A pure\footnote{By pure we mean a program that has no effects.} program is not much fun.
\end{frame}

\begin{frame}
  \frametitle{Fundamental different approaches to effects}
  \begin{description}
    \item[\alert<1->{Imperative}] Repeatedly performs implicit effects on shared global state.
    \item[\alert<1->{Functional}] Encapsulates effects in a computational context.
  \end{description}
  \uncover<2->{This talk is oriented around \emph{functional} programming with effects.}
\end{frame}

\begin{frame}
  \frametitle{Effectful computations (I)}
  \begin{align*}
    a \to b && \text{Mathematical pure function}\\
    a \to b && \text{C/C++ (impure) function}\\
    a \to b && \text{ML (impure) function}
  \end{align*}
\end{frame}

\begin{frame}
  \frametitle{Let's be explicit about effects}
  \begin{block}{Effect annotation}
    An effect annotation gives a static description of the potential run-time behaviour of a computation.
  \end{block}
  Benefits
  \begin{itemize}
    \item Serves as documentation (clarity)
    \item Compiler can apply specific optimisations
    \item Possible to reason more precisely about programs
  \end{itemize}
\end{frame}

\begin{frame}
  \frametitle{Enter the Monad}
  \begin{center}
    ``\textit{Shall I be pure or impure?}''
    \begin{figure}
      \includegraphics[scale=0.3]{figures/lambdaman.png}
      \caption{Philip Wadler aka. Lambda Man}
    \end{figure}
  \end{center}
\begin{itemize}
  \item The Essence of Functional Programming \cite{Wadler1992}
  \item The Marriage of Effects and Monads \cite{Wadler2003}
\end{itemize}
\end{frame}

\begin{frame}
  \frametitle{Effectful computations (II)}
  \begin{align*}
    a \to b && \text{Mathematical pure function}\\
    a \to b && \text{C/C++ (impure) function}\\
    a \to b && \text{ML (impure) function}\\
    \hline
    a \to \alert<2->{m}\, b && \text{Haskell impure function}
  \end{align*}
  \uncover<2->{$\alert<2->{m}$ can be considered an effect annotation}
\end{frame}

\begin{frame}
  \frametitle{Monads}
  \begin{definition}
    A monad is a triple $(m, return, bind)$ where
    \begin{itemize}
      \item $m$ is a type constructor
      \item $return : a \to m\, a$
      \item $bind   : m\, a \to (a \to m\, b) \to m\, b$
    \end{itemize}
  \end{definition}
\end{frame}

\begin{frame}
  \frametitle{Great! Many monads, many effects}
  A couple of monads and their ``effect interpretation''
  \begin{description}
    \item[\alert<1->{IO $a$}] May perform I/O, returns $a$
    \item[\alert<1->{Reader $r$ $a$}] May read from $r$, returns $a$
    \item[\alert<1->{Writer $w$ $a$}] May write to $w$, returns $a$
    \item[\alert<1->{State $s$ $a$}] May read/write some state $s$, returns $a$
    \item[\alert<1->{Maybe $a$}] May fail, returns $a$ on success
  \end{description}
\end{frame}

\begin{frame}
  \frametitle{Effectful computations (III)}
  \begin{align*}
    a &\to b && \text{Mathematical pure function}\\
    a &\to b && \text{C/C++ (impure) function}\\
    a &\to b && \text{ML (impure) function}\\
    \hline
    a &\to m_1\,m_2\, b && \text{Haskell impure function}\\
    & \not\simeq\\
    a &\to m_2\,m_1\, b
  \end{align*}
%  Sadly, monads do not compose well. Recall the signature of bind:
%  \[ \alert<1->{m}\, a \to (a \to \alert<1->{m}\, b) \to \alert<1->{m}\, b \]
%  \alert<1->{m} is fixed throughout the computation!
\end{frame}

\begin{frame}
  \frametitle{IO Monad is equivalent to a calzone pizza}
  \begin{center}   
    \begin{figure}
      \includegraphics[scale=0.9]{figures/calzone.jpg}
      \caption{``There's only meat sauce inside'', they said, but you can't really be sure.}
    \end{figure}
  \end{center}  
\end{frame}

\begin{frame}
  \frametitle{The importance of effect ordering \cite{O'Sullivan2008}}
  Two signatures\footnote{Elided details about Monad Transformers}:
  \begin{itemize}
    \item $A \defas \texttt{WriterT String Maybe}$
      \begin{itemize}
        \item Returns nothing on failure
      \end{itemize}
    \item $B \defas \texttt{MaybeT (Writer String)}$
      \begin{itemize}
        \item Returns a pair on failure
      \end{itemize}
  \end{itemize}
\end{frame}

% \begin{frame}
%   \frametitle{Problem statement}
%   \begin{block}{Problem statement}
%     \emph{How may we achieve a programming model for modular, composable and unordered effects?}
%   \end{block}
%   Proposed solution: Handlers for algebraic effects using row polymorphism!
% \end{frame}